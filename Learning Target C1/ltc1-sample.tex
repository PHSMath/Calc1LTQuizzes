\documentclass[10pt]{article}
\usepackage{palatino}
\usepackage{fullpage}
\usepackage{latexsym}
\usepackage{amsfonts}
\usepackage{hyperref}
\usepackage{multicol}
\usepackage{fancyhdr}
\usepackage{enumitem}
\usepackage{caption}
\usepackage{subcaption}
\usepackage{fancybox}
\usepackage{framed}

\pagestyle{empty}                       %no page numbers
\thispagestyle{empty}                   %removes first page number
\setlength{\parindent}{0in}

\usepackage{fullpage}
\usepackage[tmargin = 0.5in, bmargin = 1in, hmargin = 1in]{geometry}     %1-inch margins
\geometry{letterpaper}
\usepackage{graphicx}
\usepackage{amssymb}

	\thispagestyle{empty}
	\renewcommand{\headrulewidth}{0.0pt}
	\thispagestyle{fancy}
	\lhead{Name: \underline{\hspace{1.5in}}}
	\chead{MTH 201: Calculus}
	\rhead{\framebox{\textbf{Grade: \ S \quad P}} }
	\lfoot{}
	\cfoot{}
	\rfoot{}


\begin{document}

	\vspace*{0in}

		\begin{center}
			\textbf{Learning Target S8 \\
			Version 1} 
		\end{center}


\begin{framed}
	\textbf{S8: Find extreme values of a continuous function on a closed interval.}
\end{framed}

Each of items below gives a function and a closed interval. Each function is continuous on the interval given. Use calculus to find the \emph{exact} values of the absolute maximum and absolute minimum values for each function on the interval that is shown. Decimal results are not allowed --- only exact values are permitted. (For example, $\pi$ and not $3.141$.) You must use calculus; answers that are obtained using guesswork, graphs, or only numerical tables will result in a ``Progressing'' grade. 

\begin{enumerate}
    \item $p(t) = \sin(t) + \cos(t)$, $[-\frac{\pi}{2},\frac{\pi}{2}]$
    \item $q(x) = xe^{-x}$, $[0,3]$
\end{enumerate}




\vfill



\begin{small}
    \begin{framed}
        	\textbf{Criteria for Satisfactory grade:} All work is shown, and from the work an understanding of the process of finding absolute extreme values on a closed interval using calculus is apparent. At least one of the answers and the process of getting the answer must be without error; minor errors unrelated to the core concept in the other solution are OK. 
    \end{framed}

\end{small}

\end{document}
