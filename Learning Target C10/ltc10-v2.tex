\documentclass[10pt]{article}
\usepackage{palatino}
\usepackage{fullpage}
\usepackage{latexsym}
\usepackage{amsfonts}
\usepackage{hyperref}
\usepackage{multicol}
\usepackage{fancyhdr}
\usepackage{enumitem}
\usepackage{caption}
\usepackage{subcaption}
\usepackage{fancybox}
\usepackage{framed}

\pagestyle{empty}                       %no page numbers
\thispagestyle{empty}                   %removes first page number
\setlength{\parindent}{0in}

\usepackage{fullpage}
\usepackage[tmargin = 0.5in, bmargin = 1in, hmargin = 1in]{geometry}     %1-inch margins
\geometry{letterpaper}
\usepackage{graphicx}
\usepackage{amssymb}

	\thispagestyle{empty}
	\renewcommand{\headrulewidth}{0.0pt}
	\thispagestyle{fancy}
	\lhead{Name: \underline{\hspace{1.5in}}}
	\chead{MTH 201: Calculus}
	\rhead{\framebox{\textbf{Grade: \ S \quad P}} }
	\lfoot{}
	\cfoot{}
	\rfoot{}


\begin{document}

	\vspace*{0in}

		\begin{center}
			\textbf{Learning Target C10 \\
			Version 2} \\
		\end{center}

\emph{Please place all work in the blank below. If you need more room, use the back of this page.}

\begin{framed}
	\textbf{\textbf{C10:} Use the Fundamental Theorem of Calculus to compute simple integrals.}
\end{framed}

Compute the exact value of each of the following definite and indefinite integrals using the Fundamental Theorem of Calculus. Show all work, and do not use graphs or Riemann sums. Also, please leave all answers in exact form with no decimal approximations --- for example, $1/3 + \pi$ is exact while $3.4749$ is not. 


\begin{enumerate}
    \item $\displaystyle{\int_1^3 (x^2 + 1) \, dx}$
    \item $\displaystyle{\int_0^{\pi/2} \sin(t) \, dt$
    \item $\displaystyle{\int_1^2 (\sqrt{x} - x^2) \, dx}$
\end{enumerate}

\vfill


\begin{small}
    \begin{framed}
        	\textbf{Criteria for Satisfactory grade:} At least two of the integrals must be completely correct, which means: The answer is correct and in exact form, the work is shown and complete, the work is correct, and the solution uses the Fundamental Theorem of Calculus. If a third integral is incorrect, the only admissible errors are those coming from careless mistakes in algebra or arithmetic; errors in Calculus will result in a Progressing grade. 
    \end{framed}

\end{small}

\end{document}
