\documentclass[10pt]{article}
\usepackage{palatino}
\usepackage{fullpage}
\usepackage{latexsym}
\usepackage{amsfonts}
\usepackage{hyperref}
\usepackage{multicol}
\usepackage{fancyhdr}
\usepackage{enumitem}
\usepackage{caption}
\usepackage{subcaption}
\usepackage{fancybox}
\usepackage{framed}
\usepackage{amsmath}

\pagestyle{empty}                       %no page numbers
\thispagestyle{empty}                   %removes first page number
\setlength{\parindent}{0in}

\usepackage{fullpage}
\usepackage[tmargin = 0.5in, bmargin = 1in, hmargin = 1in]{geometry}     %1-inch margins
\geometry{letterpaper}
\usepackage{graphicx}
\usepackage{amssymb}

	\thispagestyle{empty}
	\renewcommand{\headrulewidth}{0.0pt}
	\thispagestyle{fancy}
	\lhead{Name: \underline{\hspace{1.5in}}}
	\chead{MTH 201: Calculus}
	\rhead{\framebox{\textbf{Grade: \ S \quad P}} }
	\lfoot{}
	\cfoot{}
	\rfoot{}


\begin{document}

	\vspace*{0in}

		\begin{center}
			\textbf{Learning Target S7 \\
			Version 2} 
		\end{center}


\begin{framed}
	\textbf{S7: Given a formula for a function, identify intervals of increase and decrease, concavity, local extrema, critical points, and inflection points.}
\end{framed}

Consider the function 
$$f(x) = \frac{1}{3}x^3 = \frac{1}{2}x^2 - 2x + 3$$
Do each of the following and show all work that pertains to your solution. You may use a calculator for numerical operations but do not use any graphs on this quiz. 
\begin{enumerate}
    \item Find $f'(x)$ and $f''(x)$. 
    \item Find all the critical numbers of $f$.
    \item Classify each critical number as a local minimum, local maximum, or neither. \emph{Show all work, including charts, that leads to your answer.} 
    \item Find the intervals on which $f$ is concave up and on which $f$ is concave down. Again, \emph{show all work, including charts, that leads to your answer.} 
    \item Find all the inflection points of $f$. 
\end{enumerate}

\vfill


\begin{small}
    \begin{framed}
        	\textbf{Criteria for Satisfactory grade:} All Calculus work is done correctly; all answers/conclusions are consistent with the calculus and algebra work done; all work that leads to the answer is shown clearly and neatly. \textbf{Please note: Insufficient or unclear work is grounds for a Progressing grade on this quiz, even if there are no mathematical errors.}
    \end{framed}

\end{small}

\end{document}
