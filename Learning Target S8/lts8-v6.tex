\documentclass[10pt]{article}
\usepackage{palatino}
\usepackage{fullpage}
\usepackage{latexsym}
\usepackage{amsfonts}
\usepackage{hyperref}
\usepackage{multicol}
\usepackage{fancyhdr}
\usepackage{enumitem}
\usepackage{caption}
\usepackage{subcaption}
\usepackage{fancybox}
\usepackage{framed}
\usepackage{amsmath}

\pagestyle{empty}                       %no page numbers
\thispagestyle{empty}                   %removes first page number
\setlength{\parindent}{0in}

\usepackage{fullpage}
\usepackage[tmargin = 0.5in, bmargin = 1in, hmargin = 1in]{geometry}     %1-inch margins
\geometry{letterpaper}
\usepackage{graphicx}
\usepackage{amssymb}

	\thispagestyle{empty}
	\renewcommand{\headrulewidth}{0.0pt}
	\thispagestyle{fancy}
	\lhead{Name: \underline{\hspace{1.5in}}}
	\chead{MTH 201: Calculus}
	\rhead{\framebox{\textbf{Grade: \ S \quad P}} }
	\lfoot{}
	\cfoot{}
	\rfoot{}


\begin{document}

	\vspace*{0in}

		\begin{center}
			\textbf{Learning Target S8 \\
			Version 6} 
		\end{center}


\begin{framed}
	\textbf{S8: Find extreme values of a continuous function on a closed interval}
\end{framed}

Find the absolute maximum and absolute minimum values of the function $f(x) = x^3 - 3x + 1$ on the closed interval $-1/2 \leq x \leq 4$. 



\vfill

\begin{small}
    \begin{framed}
        	\textbf{Criteria for Satisfactory grade:} All Calculus work is done correctly; all answers/conclusions are consistent with the calculus and algebra work done; all work that leads to the answer is shown clearly and neatly. \textbf{Please note: Insufficient or unclear work is grounds for a Progressing grade on this quiz, even if there are no mathematical errors.}
    \end{framed}

\end{small}

\end{document}
