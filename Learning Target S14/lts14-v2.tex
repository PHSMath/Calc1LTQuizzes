\documentclass[10pt]{article}
\usepackage{palatino}
\usepackage{fullpage}
\usepackage{latexsym}
\usepackage{amsfonts}
\usepackage{hyperref}
\usepackage{multicol}
\usepackage{fancyhdr}
\usepackage{enumitem}
\usepackage{caption}
\usepackage{subcaption}
\usepackage{fancybox}
\usepackage{framed}

\pagestyle{empty}                       %no page numbers
\thispagestyle{empty}                   %removes first page number
\setlength{\parindent}{0in}

\usepackage{fullpage}
\usepackage[tmargin = 0.5in, bmargin = 1in, hmargin = 1in]{geometry}     %1-inch margins
\geometry{letterpaper}
\usepackage{graphicx}
\usepackage{amssymb}

	\thispagestyle{empty}
	\renewcommand{\headrulewidth}{0.0pt}
	\thispagestyle{fancy}
	\lhead{Name: \underline{\hspace{1.5in}}}
	\chead{MTH 201: Calculus}
	\rhead{\framebox{\textbf{Grade: \ S \quad P}} }
	\lfoot{}
	\cfoot{}
	\rfoot{}


\begin{document}

	\vspace*{0in}

		\begin{center}
			\textbf{Learning Target S14} \\
			{Version 2} \\
		\end{center}

\emph{Please place all work in the blank below. If you need more room, use the back of this page.}

\begin{framed}
	\textbf{S14: Find the total amount by which a function changes on an interval using a definite integral.}
\end{framed}

A particle moves along a straight line so that its velocity at time $t$ is $v(t) = t^2 - t -6$ meters per second. Find the total change in the position of the particle during the time period $1 \leq t \leq 4$.\\

Use Calculus to find your answer and leave the answer in exact form, not a decimal approximation. Show all work, especially the Calculus steps. 



\vfill


\begin{small}
    \begin{framed}
        	\textbf{Criteria for Satisfactory grade:} The answer is in exact form, the work is shown and complete, the work is correct, and the solution uses Calculus. If the result is incorrect, the only admissible errors are those coming from careless mistakes in algebra or arithmetic; errors in Calculus will result in a Progressing grade. 
    \end{framed}

\end{small}

\end{document}
