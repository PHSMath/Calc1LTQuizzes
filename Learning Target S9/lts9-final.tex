\documentclass[10pt]{article}
\usepackage{palatino}
\usepackage{fullpage}
\usepackage{latexsym}
\usepackage{amsfonts}
\usepackage{hyperref}
\usepackage{multicol}
\usepackage{fancyhdr}
\usepackage{enumitem}
\usepackage{caption}
\usepackage{subcaption}
\usepackage{fancybox}
\usepackage{framed}
\usepackage{amsmath}

\pagestyle{empty}                       %no page numbers
\thispagestyle{empty}                   %removes first page number
\setlength{\parindent}{0in}

\usepackage{fullpage}
\usepackage[tmargin = 0.5in, bmargin = 1in, hmargin = 1in]{geometry}     %1-inch margins
\geometry{letterpaper}
\usepackage{graphicx}
\usepackage{amssymb}

	\thispagestyle{empty}
	\renewcommand{\headrulewidth}{0.0pt}
	\thispagestyle{fancy}
	\lhead{Name: \underline{\hspace{1.5in}}}
	\chead{MTH 201: Calculus}
	\rhead{\framebox{\textbf{Grade: \ S \quad P}} }
	\lfoot{}
	\cfoot{}
	\rfoot{}


\begin{document}

	\vspace*{0in}

		\begin{center}
			\textbf{Learning Target S9 \\
			Version 5} 
		\end{center}


\begin{framed}
	\textbf{S9: Set up and solve applied optimization problems.}
\end{framed}

A farmer has 10000 feet of fencing and wants to fence off a rectangular field that borders a straight river. He needs no fence along the river. What are the dimensions of the field that has the largest area?  \\

\emph{Please read the ``Criteria for Satisfactory Grade'' below for what your solution must contain in order to complete this target. A correct answer is necessary but not sufficient!} 




\vfill

\begin{small}
    \begin{framed}
        	\textbf{Criteria for Satisfactory grade:} The solution clearly states all variables involved and has a diagram when appropriate; clearly indicates which quantity is being optimized; shows all work that sets up a formula for the quantity being optimized and any constraints in the problem used toward the solution; uses Calculus to arrive at the answer, and not graphical or numerical estimates or guesses; with the exception of 1-2 minor arithmetic errors leads to a correct answer; and contains an argument explaining why the answer leads to the optimum value. Work that is incomplete, disorganized, has significant or numerous errors, and which does not include explanations for the work being done will be marked ``Unsatisfactory''. 
    \end{framed}

\end{small}

\end{document}
