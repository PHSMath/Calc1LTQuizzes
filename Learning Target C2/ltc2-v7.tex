\documentclass[10pt]{article}
\usepackage{palatino}
\usepackage{fullpage}
\usepackage{latexsym}
\usepackage{amsfonts}
\usepackage{hyperref}
\usepackage{multicol}
\usepackage{fancyhdr}
\usepackage{enumitem}
\usepackage{caption}
\usepackage{subcaption}
\usepackage{fancybox}
\usepackage{framed}

\pagestyle{empty}                       %no page numbers
\thispagestyle{empty}                   %removes first page number
\setlength{\parindent}{0in}

\usepackage{fullpage}
\usepackage[tmargin = 0.5in, bmargin = 1in, hmargin = 1in]{geometry}     %1-inch margins
\geometry{letterpaper}
\usepackage{graphicx}
\usepackage{amssymb}

	\thispagestyle{empty}
	\renewcommand{\headrulewidth}{0.0pt}
	\thispagestyle{fancy}
	\lhead{Name: \underline{\hspace{1.5in}}}
	\chead{MTH 201: Calculus}
	\rhead{\framebox{\textbf{Grade: \ S \quad P}} }
	\lfoot{}
	\cfoot{}
	\rfoot{}


\begin{document}

	\vspace*{0in}

		\begin{center}
			\textbf{Learning Target C2 \\
			Version 7} \\
			% Same as v5
		\end{center}


\begin{framed}
	\textbf{C2: Find the derivative of a function either at a point, or as a function, using the definition of the derivative.
}
\end{framed}

Consider the function $f(x) = 2x^2  + x + 5$. 

\begin{enumerate}
    \item State the limit given by the definition of the derivative that will find the value of $f'(2).$
    
    \vspace{1in} 
    
    
    \item Use algebraic techniques to evaluate the limit that you set up. You may not use graphs, numerical estimates, or techniques not studied in Chapter 1 of the text. 
\end{enumerate}



\vfill


\begin{small}
    \begin{framed}
        	\textbf{Criteria for Satisfactory grade:} The definition of the derivative is correctly stated, including the correct limit; limits are used, and only algebraic techniques for computing limits are used; there is at most 1-2 small errors, and those errors do not arise from a misunderstanding of the main concept; and all significant steps are shown. 
    \end{framed}

\end{small}

\end{document}
