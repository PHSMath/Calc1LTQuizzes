\documentclass[10pt]{article}
\usepackage{palatino}
\usepackage{fullpage}
\usepackage{latexsym}
\usepackage{amsfonts}
\usepackage{hyperref}
\usepackage{multicol}
\usepackage{fancyhdr}
\usepackage{enumitem}
\usepackage{caption}
\usepackage{subcaption}
\usepackage{fancybox}
\usepackage{framed}

\pagestyle{empty}                       %no page numbers
\thispagestyle{empty}                   %removes first page number
\setlength{\parindent}{0in}

\usepackage{fullpage}
\usepackage[tmargin = 0.5in, bmargin = 1in, hmargin = 1in]{geometry}     %1-inch margins
\geometry{letterpaper}
\usepackage{graphicx}
\usepackage{amssymb}

	\thispagestyle{empty}
	\renewcommand{\headrulewidth}{0.0pt}
	\thispagestyle{fancy}
	\lhead{Name: \underline{\hspace{1.5in}}}
	\chead{MTH 201: Calculus}
	\rhead{\framebox{\textbf{Grade: \ S \quad P}} }
	\lfoot{}
	\cfoot{}
	\rfoot{}


\begin{document}

	\vspace*{0in}

		\begin{center}
			\textbf{Learning Target S2} \\
			{Version 3} \\
		\end{center}

\emph{Please place all work in the blank below. If you need more room, use the back of this page.}

\begin{framed}
	\textbf{S2: Compute the limit of a function at a specific point using algebra.}
\end{framed}

Compute the exact value of each of the following limits using algebraic techniques. That is, do not use graphs, tables, or any kind of estimation or approximation. Show all steps in your work. If a given limit fails to exist, write ``DNE'' and then explain your reasoning. 

\begin{enumerate}
    \item $\displaystyle{\lim_{x \to -1} \frac{x^2-1}{x+1}}$
    \item $\displaystyle{\lim_{s \to 3} \frac{s^2 - 2s - 3}{s-3}$
\end{enumerate}

\vfill


\begin{small}
    \begin{framed}
        	\textbf{Criteria for Satisfactory grade:} Each limit is computed using algebra (not graphs or estimations, the algebra is correct, and the correct limit is computed. Also, all important steps in the process are shown. One or two small errors unrelated to the main concept are acceptable.
    \end{framed}

\end{small}

\end{document}
